% \documentclass[fleqn]{article}

%% Created with wxMaxima 19.10.0

\setlength{\parskip}{\medskipamount}
\setlength{\parindent}{0pt}
% \usepackage{iftex}
% \ifPDFTeX
%   % PDFLaTeX or LaTeX 
%   \usepackage[utf8]{inputenc}
%   \usepackage[T1]{fontenc}
%   \DeclareUnicodeCharacter{00B5}{\ensuremath{\mu}}
% \else
%   %  XeLaTeX or LuaLaTeX
%   \usepackage{fontspec}
% \fi
% \usepackage{graphicx}
% \usepackage{color}
% \usepackage{amsmath}
% \usepackage{grffile}
% \usepackage{ifthen}
\newsavebox{\picturebox}
\newlength{\pictureboxwidth}
\newlength{\pictureboxheight}
\newcommand{\includeimage}[1]{
    \savebox{\picturebox}{\includegraphics{#1}}
    \settoheight{\pictureboxheight}{\usebox{\picturebox}}
    \settowidth{\pictureboxwidth}{\usebox{\picturebox}}
    \ifthenelse{\lengthtest{\pictureboxwidth > .95\linewidth}}
    {
        \includegraphics[width=.95\linewidth,height=.80\textheight,keepaspectratio]{#1}
    }
    {
        \ifthenelse{\lengthtest{\pictureboxheight>.80\textheight}}
        {
            \includegraphics[width=.95\linewidth,height=.80\textheight,keepaspectratio]{#1}
            
        }
        {
            \includegraphics{#1}
        }
    }
}
\newlength{\thislabelwidth}
% \DeclareMathOperator{\abs}{abs}
% \usepackage{animate} % This package is required because the wxMaxima configuration option
                      % "Export animations to TeX" was enabled when this file was generated.

\definecolor{labelcolor}{RGB}{100,0,0}

% \begin{document}


\noindent
%%%%%%%%%%%%%%%
%%% INPUT:
\begin{minipage}[t]{4em}\color{red}\bfseries
(\% i1) 
\end{minipage}
\begin{minipage}[t]{\textwidth}\color{blue}
assume(t\ensuremath{>}=0,t\_1\ensuremath{>}0,t\_2\ensuremath{>}t\_1,t\_3\ensuremath{>}t\_2);
\end{minipage}
%%% OUTPUT:
\[\displaystyle \tag{\% o1} 
[t\mbox{>  =}0,{t_1}\mbox{>  }0,{t_2}\mbox{>  }{t_1},{t_3}\mbox{>  }{t_2}]\mbox{}
\]
%%%%%%%%%%%%%%%


\noindent
%%%%%%%%%%%%%%%
%%% INPUT:
\begin{minipage}[t]{4em}\color{red}\bfseries
(\% i4)
\end{minipage}
\begin{minipage}[t]{\textwidth}\color{blue}
V\_1(t):=V\_m/sqrt(t\_1)*sqrt(t);V\_2(t):=V\_m;V\_3(t):=V\_m*sqrt(t\_3-t)/sqrt(t\_3-t\_2);
\end{minipage}
%%% OUTPUT:
\[\displaystyle \tag{\% o2} 
{V_1}(t):=\frac{{V_m}}{\sqrt{{t_1}}} \sqrt{t}\mbox{}\]

\[\tag{\% o3} 
{V_2}(t):={V_m}\mbox{}\]

\[\tag{\% o4} 
{V_3}(t):=\frac{{V_m} \sqrt{{t_3}-t}}{\sqrt{{t_3}-{t_2}}}\mbox{}
\]
%%%%%%%%%%%%%%%


\noindent
%%%%%%%%%%%%%%%
%%% INPUT:
\begin{minipage}[t]{4em}\color{red}\bfseries
(\% i5)
\end{minipage}
\begin{minipage}[t]{\textwidth}\color{blue}
is (V\_1(0)=0 andV\_1(t\_1)=V\_2(t\_1) andV\_2(t\_1)=V\_m andV\_2(t\_2)=V\_m andV\_2(t\_1)=V\_2(t\_2) andV\_2(t\_2)=V\_3(t\_2) andV\_3(t\_3)=0);
\end{minipage}
%%% OUTPUT:
\[\displaystyle \tag{\% o5} 
\mbox{true}\mbox{}
\]
%%%%%%%%%%%%%%%


\noindent
%%%%%%%%%%%%%%%
%%% INPUT:
\begin{minipage}[t]{4em}\color{red}\bfseries
(\% i16)
\end{minipage}
\begin{minipage}[t]{\textwidth}\color{blue}
X\_1(t):=integrate(V\_1(s),s,0,t);
\end{minipage}
%%% OUTPUT:
\[\displaystyle \tag{\% o16} 
{X_1}(t):=\int_{0}^{t}{\left. {V_1}(s)ds\right.}\mbox{}
\]
%%%%%%%%%%%%%%%


\noindent
%%%%%%%%%%%%%%%
%%% INPUT:
\begin{minipage}[t]{4em}\color{red}\bfseries
(\% i17)
\end{minipage}
\begin{minipage}[t]{\textwidth}\color{blue}
X\_2(t):=X\_1(t\_1)+integrate(V\_2(s),s,t\_1,t);
\end{minipage}
%%% OUTPUT:
\[\displaystyle \tag{\% o17} 
{X_2}(t):={X_1}\left( {t_1}\right) +\int_{{t_1}}^{t}{\left. {V_2}(s)ds\right.}\mbox{}
\]
%%%%%%%%%%%%%%%


\noindent
%%%%%%%%%%%%%%%
%%% INPUT:
\begin{minipage}[t]{4em}\color{red}\bfseries
(\% i18)
\end{minipage}
\begin{minipage}[t]{\textwidth}\color{blue}
X\_3(t):=X\_2(t\_2)+integrate(V\_3(s),s,t\_2,t);
\end{minipage}
%%% OUTPUT:
\[\displaystyle \tag{\% o18} 
{X_3}(t):={X_2}\left( {t_2}\right) +\int_{{t_2}}^{t}{\left. {V_3}(s)ds\right.}\mbox{}
\]
%%%%%%%%%%%%%%%


\noindent
%%%%%%%%%%%%%%%
%%% INPUT:
\begin{minipage}[t]{4em}\color{red}\bfseries
(\% i21)
\end{minipage}
\begin{minipage}[t]{\textwidth}\color{blue}
X\_1(t);X\_2(t);X\_3(t);
\end{minipage}
%%% OUTPUT:
\mbox{}\\Is 
\[\displaystyle t\mbox{}
\] positive or zero?
\[\displaystyle positive;\mbox{}\]

\[\tag{\% o19} 
\frac{2 {V_m} {{t}^{\frac{3}{2}}}}{3 \sqrt{{t_1}}}\mbox{}\]

\[\tag{\% o20} 
\frac{2 {V_m} {t_1}}{3}+{V_m} \left( t-{t_1}\right) \mbox{}\]

\[\tag{\% o21} 
\frac{{V_m} \left( \frac{\sqrt{{t_3}-{t_2}} \left( 2 {t_3}-2 {t_2}\right) }{3}-\frac{\sqrt{{t_3}-t} \left( 2 {t_3}-2 t\right) }{3}\right) }{\sqrt{{t_3}-{t_2}}}+{V_m} \left( {t_2}-{t_1}\right) +\frac{2 {V_m} {t_1}}{3}\mbox{}
\]
%%%%%%%%%%%%%%%


\noindent
%%%%%%%%%%%%%%%
%%% INPUT:
\begin{minipage}[t]{4em}\color{red}\bfseries
(\% i22)
\end{minipage}
\begin{minipage}[t]{\textwidth}\color{blue}
factor(X\_3(t\_3));
\end{minipage}
%%% OUTPUT:
\[\displaystyle \tag{\% o22} 
\frac{{V_m} \left( 2 {t_3}+{t_2}-{t_1}\right) }{3}\mbox{}
\]
%%%%%%%%%%%%%%%
% \end{document}
