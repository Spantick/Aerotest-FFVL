%% LyX 2.3.1-1 created this file.  For more info, see http://www.lyx.org/.
%% Do not edit unless you really know what you are doing.
\documentclass[english]{article}
\usepackage[T1]{fontenc}
\usepackage[latin9]{inputenc}
\setcounter{secnumdepth}{-2}
\setcounter{tocdepth}{2}
\usepackage{babel}
\usepackage{wrapfig}
\usepackage{graphicx}
\usepackage[unicode=true]
 {hyperref}

\makeatletter

%%%%%%%%%%%%%%%%%%%%%%%%%%%%%% LyX specific LaTeX commands.
%% Because html converters don't know tabularnewline
\providecommand{\tabularnewline}{\\}
%% A simple dot to overcome graphicx limitations
\newcommand{\lyxdot}{.}


\makeatother

\begin{document}
\title{FFVL, tests structure}
\author{pierre@puiseux.name}
\date{28 octobre 2019}
\maketitle

\section{Le problème}

trois cas de rupture de suspentes on été relevés sur un parapente
biplace de marque Supair, modèle Sora2-38 (ou bien Sora2-42 ?). Ces
voiles ont subit un test structure chez \href{https://para-test.com/}{Air Turquoise}.

Ce document tente de répondre à la question : comment et pourquoi
cette voile, testée en structure, peut-elle dé-suspenter en vol ? 

Concernant la Sora2-38, aucun enregistrement des test structure, n'est
disponible sur le site d'Air Turquoise\ref{fig:Sora-2-38,-documents}. 

Concernant la Sora2-42, on trouve une représentation graphique des
enregistrements au téléchargement\ref{fig:Sora-2-42,-documents},
à la page 2 du document PG\_1429.2018\_Test\_Sora2\_42.pdf

A defaut de document précis concernant la Sora2-28, l'analyse sera
menée sur le modèle Sora2-42

\section{Premiers pas}
\begin{enumerate}
\item Pour convertir une charge de kilogrammes en Newtons : 
\[
x_{N}=x_{kg}\times9.81
\]
\item Pour convertir une vitesse de km/h en m/s :
\[
x_{m/s}=x_{km/h}\times0.28
\]
\end{enumerate}

\subsection{Les données}
\begin{itemize}
\item Air Turquoise : 
\begin{itemize}
\item le camion est un Dodge, à priori dans les 390 cv,
\item son poids à vide ($p_{c}$) est $2100\ kg$, soit $p_{c}=20594\ N$
, le poids max en charge $3000\ kg$. 
\item Il atteint des charges appliquées à un biplace ($RFA$) de $1700$
à $2000\ kg$, soit 
\[
1733\ N\leq RFA\leq2039\ N
\]
\item On suppose que Air Turquoise parvient aux mêmes vitesses que Aérotest,
soit 
\[
V_{max}=28\ m/s
\]
\item Longueur de la piste : 
\[
L=850\ m
\]
\end{itemize}
\item Aérotest : 
\begin{itemize}
\item les vitesses que nous atteignons régulièrement se situent autour de
$100\pm10\ km/h$, soit 
\[
V_{max}=28\ m/s
\]
\item Aérotest : parvient à des charges appliquées à un biplace de $2500\ kg$
soit 
\[
RFA=24525\ N
\]
\item la piste que nous empruntons en fait 
\[
L=1700\ m
\]
\end{itemize}
\end{itemize}

\subsection{Première analyse}

Dans les deux cas (Aérotest et Air Turquoise), les graphiques de montée
en charge ont une tendance linéaire marquée. Sur cette plage linéaire,
on peut supposer que l'accélération est également constante(??). Dans
ce cas, elle vaut ($\Delta L$ est la distance de roulage, on suppose
que c'est la longueur de la piste moins $50\ m$)
\[
\gamma=\frac{\Delta L}{\left(\Delta t\right)^{2}}
\]

\begin{tabular}{llll}
\hline 
Air Turquoise & BiGolden 4 light & Windtech Ru-Bi 2 & Super-Sora 2(42)\tabularnewline
\hline 
\noalign{\vskip0.3cm}
$\Delta F\ \left(N\right)$ & 16677 & 17658.0 & 16677.0\tabularnewline
$\Delta t\ \left(s)\right)$ & 30 & 30 & 31\tabularnewline
$\frac{\Delta F}{\Delta t}\ \left(N/s\right)$ & 555.9 & 588.6 & 538.0\tabularnewline
$\gamma\ \left(m/s^{2}\right)$ & 0.89 & 0.89 & \tabularnewline
\hline 
\end{tabular}

\begin{tabular}{|l|l|l|l|l|}
\hline
\noalign{\vskip0.1cm}
Modèle & $\Delta F\ \left(N\right)$ & $\frac{\Delta F}{\Delta t}\ \left(N/s\right)$ & $\gamma \left(m/s^2\right)$ & Remarque \tabularnewline
\noalign{\vskip0.1cm}
\hline
Hercules & 22840 & 50.0  & 1600.0 & 5 pics à 2487 N \tabularnewline
MacPara Trike 42 & 20850 & 48.0  & 1600.0 & 5 pics à 2233 N \tabularnewline
?? & 21092 & 35.0  & 1600.0 & ?? \tabularnewline
\hline
\end{tabular}

si on suppose que l'accélération est, comme la montée en charge, constante,
elle vaut $\gamma=\frac{\Delta V}{\Delta t}$. Dans les deux cas, 

\part{Annexes}

\subsection{Voile de référence}

\begin{wrapfigure}{o}{0.5\columnwidth}%
\includegraphics[width=0.8\textwidth]{\string"Air Turquoise/Capture d'écran 2019-10-28 04.49.26\string".png}

\caption{Sora 2-38, les documents disponibles, absence de tests structure\label{fig:Sora-2-38,-documents}}
\end{wrapfigure}%

\begin{figure}
\includegraphics[width=0.8\textwidth]{\string"Air Turquoise/Capture d'écran 2019-10-28 05.24.59\string".png}

\caption{Sora 2-38, les documents disponibles, absence de tests structure\label{fig:Sora-2-42,-documents}}
\end{figure}

\end{document}
